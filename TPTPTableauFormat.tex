%----Rules for writing with Geoff
% 1. No second person, unless you are serious saying "this is what __we__ did. Rather use third
% person passive, e.g., instead of "we did this" use "this was done".
% 2. Start each sentence on a new line, to make it easy to reorder sentences.
% 3. Send me your BibTeX entries - I'll add them to the bibliography file in my standard format.
%----
\documentclass[runningheads]{llncs}

\usepackage[T1]{fontenc}
\usepackage{graphicx}
% \renewcommand\UrlFont{\color{blue}\rmfamily}
\usepackage{hyperref}
\usepackage{color}
\usepackage{setspace}
\usepackage{verbatim}
\usepackage{multicol}
\usepackage{array}
\usepackage{bbding}
\newcommand{\PreserveBackslash}[1]{\let\temp=\\#1\let\\=\temp}
\newcolumntype{C}[1]{>{\PreserveBackslash\centering}p{#1}}
\newcolumntype{R}[1]{>{\PreserveBackslash\raggedleft}p{#1}}
\newcolumntype{L}[1]{>{\PreserveBackslash\raggedright}p{#1}}
\setlength{\tabcolsep}{3pt}

\renewcommand{\floatpagefraction}{0.9}
\renewcommand{\textfloatsep}{2.0ex}
\renewcommand{\dbltextfloatsep}{2.0ex}

\newenvironment{packed_itemize}{
\vspace*{-0.5em}
\begin{itemize}
\setlength{\partopsep}{0pt}
\setlength{\itemsep}{1pt}
\setlength{\parskip}{0pt}
\setlength{\parsep}{0pt}
}{\end{itemize}}

\begin{document}

\title{The TPTP Format for Tableaux Proofs}
\titlerunning{TPTP Tableaux}

\author{
Geoff Sutcliffe\inst{1}\orcidID{0000-0001-9120-3927}\Envelope
\and
\\ Sean Holden\inst{2}\orcidID{0000-0000-0000-0000}
\and
\\ Mantas Baksys\inst{2}\orcidID{0000-0001-9532-1007}
}
\authorrunning{Geoff Sutcliffe, et al.}
\institute{University of Miami, Miami, USA,
\email{geoff@cs.miami.edu},
\and
University of Cambridge, Cambridge, United Kingdom,
\email{sbh11@cl.cam.ac.uk,mb2412@cam.ac.uk}
}

\maketitle
%--------------------------------------------------------------------------------------------------
\begin{abstract}
Geoff

\keywords{Geoff}
\end{abstract}
%--------------------------------------------------------------------------------------------------
\section{Introduction}
\label{Introduction}

Geoff

\paragraph{Related work:}

Sections~\ref{BLAH} provide ...
Section~\ref{Conclusion} concludes.

%--------------------------------------------------------------------------------------------------
\section{The TPTP World}
\label{TPTP}

Geoff

%--------------------------------------------------------------------------------------------------
\subsection{The TPTP Language}
\label{TPTPLanguage}

Geoff

%--------------------------------------------------------------------------------------------------
\subsection{The TPTP Format for Derivations}
\label{Derivations}

Geoff. Include GDV here.

%--------------------------------------------------------------------------------------------------
\section{The (new) TPTP Format for Tableau Proofs}
\label{Tableau}

Geoff: Requirements: semantic verification, tableau reconstruction.

Mantas: Explain the format, with two examples.

% \begin{figure}[t!]
% \centering
% \includegraphics[width=0.6\textwidth]{Plots/FOF_THM_RFO_PPP/FOF_THM_RFO_PPP}
% \vspace*{-1em}
% \caption{CPU times for {\tt FOF\_THM\_RFO\_*}}
% \label{PPPPlot}
% \end{figure}

%--------------------------------------------------------------------------------------------------
\section{Tableau ATP Systems}
\label{ATPSystems}

Sean.

Connect++ and more.

Results from verifying a bunch of Connect++ tableau if we get the software done.

%--------------------------------------------------------------------------------------------------
\section{Conclusion}
\label{Conclusion}

All.

%--------------------------------------------------------------------------------------------------
\bibliographystyle{splncs04}
\bibliography{Bibliography}
%--------------------------------------------------------------------------------------------------
\end{document}
